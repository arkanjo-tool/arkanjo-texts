\en

The Linux kernel is a widely used Free/Libre/Open Source Software (FLOSS) project,
powering the majority of webservers and smartphones worldwide.
Maintaining the kernel is an enormous task, involving more than 28 million lines of 
code
\footnote{Data directly extracted from Linux github repository~\citep{linuxrepo}.
Counted  28783390 lines of code and 20699 contributors on May 29th, 2025.
Number of contributors counted as unique full names listed as a patch author.}
and over 2,000 developers are actively contributing to the kernel
\footnote{Data from \url{https://lwn.net/Articles/1022414/}}.
Due to the scale of the project, not all contributions adhere to best practices, resulting 
in poor-quality code artifacts that may complicate maintenance and future feature development.

One such artifact is code duplication, which is the focus of this research. The motivation for 
this work began with a practical problem rooted in device drivers, which are a significant part 
of the kernel, representing 66\% of the source code~\citep{marcelo}. We interacted with the 
maintainers of the AMD Display driver, an essential component for the 19\% of personal GPUs 
manufactured by AMD in 2023~\citep{gpumarket}. They shared their challenges with a significant 
amount of duplicated code that hampers the driver maintenance. In searching for a solution, 
we found that existing tools are not well-suited for identifying duplications in large-scale 
codebases like the Linux kernel, nor do they offer guidance on resolving the duplications 
they find. Despite searching formal and grey literature, existing solutions failed to address
these specific challenges. The list of related tools examined in this exploratory 
research is available in Appendix~\ref{app:gray}.

This work presents an approach to address the code duplication issue in the Linux kernel; to 
support this approach, we developed ArKanjo, a command-line tool designed to detect and analyze 
function-level duplications. Released under the MIT license, ArKanjo uses a two-stage 
architecture with a Preprocessor and a Query Responder. This design separates the heavy 
computational analysis from the fast querying of duplication results, making it effective for 
large codebases. We validated the tool by comparing its results with the BigCloneBench
dataset~\citep{bigclonebench} and by conducting an empirical analysis on a randomly selected set of 
files from the AMD Display driver.

Beyond detection, we also investigated if the duplications found by ArKanjo could be mitigated.
This was achieved through an ethnographic approach, which included a participant observation 
study where we attempted to mitigate duplications, and a non-participant observation study of 
university students tasked with fixing duplications found by the tool. These studies showed that 
ArKanjo can effectively lower the barrier for new contributors, as evidenced by its role in 
guiding students to make their first code contributions to the kernel.

\section**{Research Design}

\label{sec:intresearch}

To address the code duplication problem within the Linux kernel, this research employs a 
multi-method approach to bridge the gap between detection and practical mitigation. 
Our approach, illustrated in Figure~\ref{fig:reDesign}, is structured into two interconnected phases that progress 
from tool development to an in-depth study of its application within the kernel development community. 
The overall objective is to produce a functional tool and actionable insights into 
the patterns and practices surrounding code duplication and its resolution.

\begin{figure}[h]
\includegraphics[scale=0.9]{research_design}
\caption{Diagram of the research design.}
\label{fig:reDesign}
\end{figure}

The initial phase of our research focused on developing and validating ArKanjo, a
command-line tool capable of identifying code duplication at the function level. 
We rigorously validated the tool accuracy by evaluating its performance against the
standard BigCloneBench dataset~\citep{bigclonebench} and by conducting an empirical analysis
of its results within the AMD Display driver. This foundational work ensured we had a
reliable instrument to proceed with the practical investigation of code duplication
in our target subsystems.

With a validated tool, the research moved from detection to practice, investigating the 
real-world challenges of mitigating the identified duplications. This was accomplished 
through an ethnographic study to engage with the kernel community and understand their 
perspectives on code quality and refactoring. This study had two components: a 
participant-observation experiment where the author submitted refactoring patches to the 
AMD Display driver, and a non-participant observation of university students using ArKanjo
to contribute fixes to both the AMD Display and Industrial I/O subsystems.

By combining tool 
development with direct community interaction, this research design allows for a comprehensive 
understanding of the technical and social factors that influence the mitigation of duplicated 
code in a large-scale FLOSS project.

\section**{Thesis Structure}

This manuscript consists of five more chapters. Chapter~\ref{cha:back} presents
the literature overview for code duplication detection (main definitions,
current approaches in the literature), a brief description of the Linux kernel and the components
explored in this work, 
and a review of refactoring methods used throughout this research.
Chapter~\ref{cha:tool} presents ArKanjo, our proposed tool to detect code duplication, 
describing all the main components. Chapter~\ref{cha:method}
describes the research methods selected to guide our work.
Chapter~\ref{cha:results} shows the results we had through our work, from the research methods to 
evaluate our tool and the ethnographic studies. Chapter~\ref{cha:conclusion} concludes this research.
