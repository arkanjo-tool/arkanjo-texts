\begin{table}
\begin{tabular}{ | c | c | c | c | c | c | m{8em} | }

\hline

\textbf{Similarity range} & \textbf{T1} & \textbf{T2} & \textbf{T3} & \textbf{T4}
& \textbf{Negative} & \textbf{Percentage} \\ \hline 
99\% - 100\% & 9 & 0 & 0 & 1 & 0 & \textbf{100\%} \\ \hline
89\% - 91\% & 0 & 8 & 0 & 1 & 1 & \textbf{90\%} \\ \hline
79\% - 81\% & 0 & 3 & 2 & 3 & 2 & \textbf{80\%} \\ \hline
69\% - 71\% & 0 & 3 & 1 & 1 & 5 & \textbf{50\%} \\ \hline
59\% - 61\% & 0 & 0 & 0 & 2 & 8 & \textbf{20\%} \\ \hline
49\% - 51\% & 0 & 0 & 0 & 0 & 0 & \textbf{0\%} \\ \hline
39\% - 41\% & 0 & 0 & 0 & 0 & 0 & \textbf{0\%} \\ \hline
29\% - 31\% & 0 & 0 & 0 & 0 & 10 & \textbf{0\%} \\ \hline

\hline

\end{tabular}
\label{tab:emp}
\caption{Results of the proposed tool in the empirical analyses in the AMD Display driver}

\end{table}
