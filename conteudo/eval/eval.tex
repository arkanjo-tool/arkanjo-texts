On this Chapter, we exposes our experiment results bout the proposed tool performance following
the methology described in Chapter \ref{cha:method}. Firstly, we shows the results obtained by our tool
in the BigCloneBench dataset\citep{bigclonebench}, which is the standard form of the formal literature 
\citep{litreview}. Secondly, we shows the results obtained by our tool on the empirical analyses. 
Finaly, we do a triangulation of data analysing our finds on the tool performance.

\subsection{Tool results on the BigCloneBench dataset}

\begin{table}
\begin{tabular}{ | m{6em} | c | c | c | c | c | c | }

\hline

\textbf{Minimum Similarity Threshould} & \textbf{T1} & \textbf{T2} & \textbf{ST3} & \textbf{MT3}
& \textbf{WT3/T4} & \textbf{Negative} \\ \hline 

100\% & 100\% & 5\% & 6\% & 0\% & 0\% & 100\% \\ \hline
90\% & 100\% & 85\% & 26\% & 0\% & 0\% & 100\% \\ \hline
80\% & 100\% & 87\% & 37\% & 1\% & 0\% & 100\% \\ \hline
70\% & 100\% & 88\% & 44\% & 2\% & 0\% & 100\% \\ \hline
60\% & 100\% & 89\% & 49\% & 4\% & 0\% & 100\% \\ \hline
50\% & 100\% & 90\% & 64\% & 6\% & 0\% & 100\% \\ \hline
40\% & 100\% & 90\% & 68\% & 9\% & 0\% & 100\% \\ \hline
30\% & 100\% & 98\% & 74\% & 13\% & 0\% & 100\% \\ \hline

\hline

\end{tabular}
\label{tab:bigclone}
\caption{Recall of the proposed tool in the BigCloneBench dataset varying minimum similarity threshould.}
\end{table}


As described in section \ref{sec:metbig}. For the evaluation of our proposed tool in the
BigCLoneBench dataset  \citep{bigclonebench}, we define that every pair of functions in which the 
similarity metric is equal or superior to a threshold X will be considered a duplication, and the 
pair of code files in which the functions exist are labeled as duplication. If, on this method, 
a code duplication pair is correctly identified, it will count as correct to its code duplication 
types. If an incorrect duplication pair is inferred as a duplication, it will count as incorrect 
to all code duplication types. Table \ref{tab:bigclone} shows the results obtained by our tool in the 
BigCloneBench dataset \citep{bigclonebench}.

The results presented in Table \ref{tab:bigclone} indicate that our tool is a pessimist. There 
is no negative sample inferred as duplication even for a minimum similarity threshold of 30\%, 
on the other hand, it is bad for identifying code duplications of complex code duplication types, 
that is Medium Type-3 and Weak Type-3/Type-4. As mentioned in Chapter \ref{cha:tool}, if a 
better result is required on these complex code duplication types, it is possible to change 
the code duplication detection method, as it is designed to be a replaceable component in our 
tool, given the limitations.

\subsection{Tool results on the empirical analyses}

As described in section \ref{sec:metemp} For the evaluation of our proposed tool by our empirical 
analyses, we randomly selected ten function pairs for a variation of ranges of the similarity 
metric computed by our tool. After that, we analyzed if they were duplications, and if yes, 
we judged the duplication type. Table \ref{tab:emp} presents the results obtained by our tool 
in the empirical analyses.

\begin{table}
\begin{tabular}{ | c | c | c | c | c | c | m{8em} | }

\hline

\textbf{Similarity range} & \textbf{T1} & \textbf{T2} & T3 & T4
& \textbf{Negative} & \textbf{Percentage} \\ \hline 
99\% - 100\% & 9 & 0 & 0 & 1 & 0 & \textbf{100\%} \\ \hline
89\% - 91\% & 0 & 8 & 0 & 1 & 1 & \textbf{90\%} \\ \hline
79\% - 81\% & 0 & 3 & 2 & 3 & 2 & \textbf{80\%} \\ \hline
69\% - 71\% & 0 & 3 & 1 & 1 & 5 & \textbf{50\%} \\ \hline
59\% - 61\% & 0 & 0 & 0 & 2 & 8 & \textbf{20\%} \\ \hline
49\% - 51\% & 0 & 0 & 0 & 0 & 0 & \textbf{0\%} \\ \hline
39\% - 41\% & 0 & 0 & 0 & 0 & 0 & \textbf{0\%} \\ \hline
29\% - 31\% & 0 & 0 & 0 & 0 & 10 & \textbf{0\%} \\ \hline

\hline

\end{tabular}
\caption{Results of the Arkanjo tool in the empirical analyses in the AMD Display driver}
\label{tab:emp}

The results were obtained by randomly selecting ten functions in the AMD Display driver for 
each similarity range and analyzing if they are duplications and which duplication type.

\end{table}


The results presented in Table \ref{tab:emp} indicate that as the similarity metric reduces, 
there is a significant increase in function pairs that are not duplications in the Linux kernel. 
For the ranges 79\%-81\% of similarity or greater, we have a high percentage of code duplications. 
With 69\%-71\% of similarity, the rate goes down to 50\%. With 59\%-61\%, the rate decreases to 20\%.
With 49\%-51\% similarity or lower, we did not find any code duplications.

\subsection{Triangulation of results}

Both the BigCloneBench and the empirical method results point out that our tool has a good 
result for detecting Type-1 and Type-2 code clone duplications and has a poor performance 
in detecting the other types. Using a similarity metric between 80\% and 100\% shows good 
results in both methods.

There is a discrepancy in the results of non-code duplications inferred as duplications 
between the two methods. In the BigCloneBench results, we did not find any negative samples,
while in the empirical analyses, we found a higher percentage of negative samples. We have 
two possible reasons for this discrepancy.

The first reason is the limitations and problems of the BigCloneBench presented by Krinke 
and Ragkhitwetsagul \citep{bigfail}. The second reason is that the AMD Display driver 
is a system where each code artifact shares a meaning. While the BigCloneBench is a 
collection of self-contained code artifacts without sharing meaning with each, as our 
tool uses a text-based code clone detection method, it is expected to have poorer 
performance in the AMD Display driver than in the BigCloneBench dataset. We do not know 
if the state-of-the-art code duplication detection has the same problem and is obscured 
by the evaluation in the BigCloneBench dataset.
