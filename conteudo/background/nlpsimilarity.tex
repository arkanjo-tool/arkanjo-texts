\en

\section{Text similarity detection methods}
\label{sec:similarity}.

The standard methodology for text similarity detection is grounded on building vector embeddings of 
the texts and applying a distance function in these embeddings to measure the similarity between them. 
There are multiple vector embedding methods proposed by the literature, such as bag-of-words, TF-IDF, 
LSI, and Word2Vec \citep{gensimlivro}.
Recent works related to building vector embeddings use Large Language Models (LLM), for example,
the work of Ethayarajh, which uses BERT, ELMo, and GPT \citep{llmsimilar}. 
The most popular distance function used by the literature is the cosine similarity defined by the formula
\citep{cosineref}:

$$\text{cosine similarity} = SC(A,B) = \frac{ A \cdot B}{ \lVert A \rVert \lVert B \rVert }$$

Where $A$ and $B$ are the vector embeddings of the two texts, an advantage of the cosine similarity is that $SC(A,B)$ is
a number between $0$ and $1$, where $SC(A,B)$ values closer to the number $1$ means that two vectors are more similar, 
it is usual to compare the cosine similarity to approximation to a probability value, transforming the values between 
$0$ and $1$ to a percentage between $0 \% $ and $100\%$, through our research, we will use the percentage form.

It is a famous fact that LLM has a high computational cost. An example of the high computational cost is an analysis 
done by Reimers and Gurevych to find the most similar pair in a collection of 10,000 sentences, 
which required about 65 hours with BERT \citep{bertsimilar}. 
We want to compare a similar magnitude of pairs in our work, so we decided not to use LLM for our vector embedding 
method, even though it is a valid solution for smaller codebases.

\subsection{Gensim}

Gensim is an open-source Python library \citep{gensim} which the owners declare themselves as "The fastest library 
for training of vector embeddings - Python or otherwise. The core algorithms in Gensim use battle-hardened, highly 
optimized and parallelized C routines." \citep{gensimsite}. For the velocity of Gensim, we chose to use this library 
for our vector embedding method.

In this work, we will not explore more alternatives for vector embeddings. We do not want to optimize the
computational cost or the code clone detection to the state of the art as they are not the main objectives 
of our research.
