\en

\section{The Linux Kernel}

The Linux kernel is a Unix-like operating system published by Linus Torvalds in 1992 as a free software, and
it started becoming popular in the late 1990s \citep{linuxbook}. Against popular belief, Linux is not a full 
operating system as it does not include applications such as filesystem utilities or windowing systems. The "Linux" 
distributions known to the public are, in fact, GNU/Linux operating systems \citep{gnuref}.

The Linux kernel is monolithic and organized as a set of subsystems such as process scheduler, memory management, 
device driver infrastructure, networking, and filesystems \citep{melissa}.
Each subsystem usually has a maintainer or a group of maintainers.  Maintainers are people responsible 
for managing and accepting contributions to the subsystem they care for \citep{melissa}.

The contributions to Linux are coordinated with the usage of the Git source management tool.
The contributions to the project are formatted as patches, which are text documents describing differences 
between two different versions of a source tree The patches are sent and reviewed through the 
Linux mailing lists \citep{melissa}.

Mailing lists are the official means of communication of the Linux community, through the mailing lists 
the community members send and review contributions to the kernel, discuss topics related to Linux such as 
which new features to add, et cetera. As there are multiple subsystems in the Linux kernel, there are 
multiple mailing lists in the Linux to organize the discussion made. Every subsystem has its mailing 
list to concentrate the discussion related to the subsystem \citep{melissa}. 
The interactions made in the mailing lists are documented and stored in the Linux kernel Lore
\citep{linuxlore}, which serves as a repository of the discussion made in the Linux kernel to be
consulted by anyone interested.

One of the Linux kernel subsystems is the AMD Display Driver, a subsystem responsible for implementing 
the drivers required to enable AMD GPU to work correctly in the Linux environment. The maintainers of 
this subsystem reported an empirical problem they have in the subsystem, which is a significant amount 
of duplicated code in the subsystem, which hinders the maintenance of the subsystem. The reported issue 
is the starting point that initiated our research, making the AMD Display Driver our principal object 
of study to understand how to mitigate duplicated code in the Linux kernel. The AMD Display Driver 
contains X code lines, Y functions/structs, and Z code files, representing a significant amount of 
the Linux kernel. As with any other Linux subsystem, the discussions related to the AMD Display Driver 
are made in its proper mailing list (FIND MAILING LIST AND CITE HERE).
