\en

\section{The Linux Kernel}

The Linux kernel is an Unix-like operating system published by Linus Torvalds in 1992 as a free software and
it started becoming popular in late 1990s \citep{linuxbook}. Against popular belief, linux is not a full 
operating system as it do not includes applications such as filesystem utilities, windowing systems. The "Linux" 
distributions know to the public are in fact GNU/Linux operating systems \citep{gnuref}.

The Linux kernel is monolithic organized as a set of subsystem such as process scheduler, memory management, 
device driver infrastruture, networking and filesystems \citep{melissa}.
Each subsystem usually have a maintainer or a group of maintainers, which are people responsible 
for managing and accepting contributions into the subsystem they care of \citep{melissa}. 

The contributions to the Linux are coordinate with the usage of the Git source management tool. The contributions 
made to the project are formated as patches, which are text documents describing differences between two different 
versions of a source tree \citep{melissa}.

AMD Display Driver is a linux's subsystem responsible for the management of AMD GPU's. This subsystem have a problem 
with code clone duplication, which are acknowledged by the subsystem's maintainters. For this reason, we selected this
subsystem as the primary focus for refactoring the detected code clone duplication by our tool.
(MAYBE CHANGE TO: For this reason, we selected this subsystem as the primary focus for our empirical analysis in our research).
