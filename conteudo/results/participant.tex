\en

\section{Participant Observation Experiment}

For our initial participant observation, we attempted to mitigate two code duplications found by ArKanjo tool
in the AMD Display driver applying our systematic approach presented in \ref{subsubsec:systematic}. 
As presented in section \ref{sec:meteth}, we opted for applying the systematic approach in the functions 
\textit{offset\_to\_id} and \textit{phy\_id\_to\_atom} in the folders \textit{dc/gpio} and \text{dc/bios} respectively.

Hereafter, the mitigation on \textit{offset\_to\_id} will be referred to as mitigation 1, and the mitigation on 
\textit{phy\_id\_to\_atom} will be referred to as mitigation 2. Table \ref{tab:patch} summarizes the impact of the mitigations
in terms of files and lines changed, and the refactoring method used. We present our experience and learning for each mitigation 
in the next sections.

\begin{table}
\begin{tabular}{ | c | c | c | c | m{6em} | }

\hline

\textbf{Mitigation} & \textbf{Files changed} & \textbf{Lines added} & \textbf{Lines removed} & \textbf{Refactoring methods used}
\\ \hline 

1 & 13 & +224 & -753 & Parameterize Method, Extract Method  \\ \hline
2 & 8 & +132 & -517 & None \\ \hline

\hline
\end{tabular}
\caption{Refactor methods used and metrics of the mitigation in the participant-observation experiment.}
\label{tab:patch}
\end{table}


\subsection{Mitigation 1}

On the process of the systematic approach, we found that the \textit{offset\_to\_id} function exist in 10 files, enclosing
multiple GPU architectures. The duplicated functions are not exactly equals, as there are some minimal specific logics applied
to each GPU architecture. To address this issue, we applied the extract method to break the function in smaller parts and the 
parameterize method to address the specific logics for the GPU's.

The base idea on how we would mitigate the duplications found were done, but we could not complete the refactoring on a state 
capable to submit as a patch to the driver. 
We found that the configuration files on the AMD Display driver are designed to be imported at compilation time
using \textit{\#define} macros. The configuration files design choice makes the refactoring of duplications on generic 
approachs trickier, as refactoring code that depends on these files requires significant refactor on the configuration files
design and deep knowledge of the codebase, which us as first-timer contributors do not have. Thus, we opted to not continue 
on investigating this mitigation.

\subsection{Mitigation 2}


\subsection{OLD}

In the first function pair, completing the refactoring requires the expertise of a maintainer for the AMD Display driver. This observation is because duplicated code artifacts involve copy-pasted code that imports different configuration files. The design of the AMD Display driver configuration files necessitates importing specific configurations at compile time, which prevents a generic approach. Consequently, refactoring this pair requires changes in how the AMD Display driver manages configuration files -- a complex task that demands input from an expert or maintainer.

The refactoring of the second function pair, however, was successful. We submitted this refactoring as a patch to the AMD Display driver to obtain feedback from the maintainers. The process of sending the patch took more time than expected. We initially submitted the patch on August 9th, 2024, but we did not receive replies. As a result, we had to resend it on October 9th, 2024 and received an initial response on November 3rd, 2024. 

We understood new details about the driver evolution processes in discussions with our partner from the AMD Display driver. Some of the code in the driver is shared across all operating systems that support the implemented GPUs. This fact creates a complex process within AMD to format and submit changes across the supported systems while simultaneously implementing measures to mitigate errors.

Additionally, we comprehended that AMD Display driver developers can not view duplicated code negatively, as we understand it is a good practice in software engineering. One example shared with us is that duplicated code enhances the independence of GPU driver code, allowing developers to make changes to a specific GPU without needing to test compatibility with others. This approach helps save significant time and effort. 

Regarding the patch content, only minor changes were requested to align with code 
style and best practices that were initially unknown to us. More details of the 
patch can be found in Appendix \ref{app:feedback}.

