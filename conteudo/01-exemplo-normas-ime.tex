%!TeX root=../tese.tex
%("dica" para o editor de texto: este arquivo é parte de um documento maior)
% para saber mais: https://tex.stackexchange.com/q/78101

\en

\chapter{Background}

\section{Code Duplication Detection}

Code Duplication Detection is a field in computer science studies that gets the attention of researchers back to 1988 \citep{firstman}.
The occurrence of Code Duplication is a harmful artifact to have in a software, affecting software's related tasks such as code readability,
introdution of bugs, addition of co-changed requirement, etc.   \citep{harmone}. 
On most of the cases, the occurrence of Code Duplication tends to create a more unstable software then the nonduplicate code counterpart   \citep{harmtwo}.

Through the years, the subject of studies about Code Duplication Detection branched out mainly in two fields of study,
Code Clone Detection and Code Plagiarism Detection,
with the former one focused on the technical aspect while the later one also introduces an social aspect on the field of research
\citep{litreview}. For our research we are only interested on the Code Clone Detection branch given the nature of our work.



\subsection{Types of Code Duplication}

The categorization of two code artifacts as a duplication of each other is a subjective task (REFERENCE HERE). 
To mitigate this human aspect on the given problem, the literature classified Code Duplication on the scale of changes in between 
the code artifacts in four main types of Code Duplication \citep{litreview}. Given two code artifacts F and G, 
the types of Code Duplication are described as follow \citep{litreview}:

\begin{itemize}
	\item \textbf{Type-1:} The differences between F and G are in the context of revision of comments, variable names, white spaces and any other kind of irrelevant elements. (ADD FIGURE) shows a typical example of Type-1 Code Duplication \citep{litreview}. 

	\item \textbf{Type-2:} The differences between F and G is the same as the Type-1, with the addition of considering addition and deletion of redudant codes. (ADD FIGURE) shows a typical example of Type-2 Code Duplication \citep{litreview}. 

	\item \textbf{Type-3:} The difference between F and G is the same as the Type-2, with the addition of considering the reorder of code blocks as well as statements within code blocks. (ADD FIGURE) shows a typical example of Type-3 Code Duplication \citep{litreview}. 

	\item \textbf{Type-4:} The difference between F and G is the same as the Type-3, with the addtionf of considering changes on data sctrucure, the order of operands/operators in expresssions, or replacing part of codes with equivalent composition. (ADD FIGURE) shows a typical example of Type-4 Code Duplication \citep{litreview}. 
\end{itemize}








\index{Tese/Dissertação!versões}

