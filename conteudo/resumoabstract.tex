%!TeX root=../tese.tex
%("dica" para o editor de texto: este arquivo é parte de um documento maior)
% para saber mais: https://tex.stackexchange.com/q/78101

% As palavras-chave são obrigatórias, em português e em inglês, e devem ser
% definidas antes do resumo/abstract. Acrescente quantas forem necessárias.
\palavraschave{Linux, Duplicação de código, ferramenta}

\keywords{Linux,Code duplication,Tool}

% O resumo é obrigatório, em português e inglês. Estes comandos também
% geram automaticamente a referência para o próprio documento, conforme
% as normas sugeridas da USP.
\resumo{

O kernel Linux é um projeto de software livre amplamente utilizado, alimentando 96,3\% 
dos um milhão maiores servidores web e 85\% dos smartphones em todo o mundo. Dada a sua extensa 
base de usuários, qualquer adição de recursos, correção de bugs ou tratamento de 
vulnerabilidades de segurança pode impactar milhões de pessoas. Manter o kernel é 
uma tarefa enorme, envolvendo mais de 19 milhões de linhas de código e contribuições 
de mais de mil desenvolvedores. Devido à escala do projeto, nem todas as contribuições 
seguem as melhores práticas, resultando em artefatos de código de baixa qualidade que 
podem complicar a manutenção e o desenvolvimento de novos recursos. Um desses artefatos 
é a duplicação de código, que é o foco desta pesquisa. As ferramentas existentes 
para detectar duplicação de código normalmente comparam dois artefatos de código para 
determinar se são duplicatas. No entanto, essas ferramentas não são adequadas para 
identificar duplicações em bases de código em grande escala, como o kernel Linux, nem 
oferecem orientações sobre como resolver as duplicações detectadas. Apesar das 
pesquisas tanto na literatura formal quanto na cinza, as soluções existentes não 
conseguem resolver os desafios específicos do kernel Linux. 
Esta pesquisa apresenta o ArKanjo, uma ferramenta de linha de comando para a manutenção 
do kernel Linux, projetada para detectar e analisar duplicações em nível de função. 
Lançado sob a licença MIT, o ArKanjo emprega uma arquitetura de dois estágios que consiste 
em um Preprocessor e um Query Responder, que separa a análise computacionalmente intensiva 
da consulta eficiente por duplicações em grandes bases de código. Em segundo lugar, ele 
analisa as duplicações identificadas para extrair padrões e métodos de refatoração, oferecendo 
orientação a futuros contribuidores. Avaliamos o ArKanjo em casos de duplicação do mundo 
real em versões recentes do kernel, demonstrando sua eficácia na identificação de clones 
problemáticos que ferramentas genéricas frequentemente ignoram. Ao identificar instâncias 
de duplicação bem definidas e gerenciáveis, o ArKanjo reduz efetivamente a barreira para 
novos contribuidores, uma capacidade evidenciada por seu papel em orientar estudantes a 
fazerem suas primeiras melhorias de código no kernel.

}

\abstract{

The Linux kernel is a widely used Free/Libre/Open Source Software (FLOSS) project, powering
96.3\% of the top one million web servers 
and 85\% of smartphones worldwide. Given its extensive user base, any addition of features, 
bug fixes, or handling of security vulnerabilities can impact millions. Maintaining the kernel 
is an enormous task, involving more than 28 million lines of code and contributions from over 
twenty thousand developers. Due to the scale of the project, not all contributions adhere to best 
practices, resulting in poor-quality code artifacts that may complicate maintenance and feature 
development. One such artifact is code duplication, which is the focus of this research. 
Existing tools for detecting code duplication typically compare two code artifacts to determine 
if they are duplicates. However, these tools are not well-suited for identifying duplications 
in large-scale codebases like the Linux kernel, nor do they guide on resolving detected 
duplications. Despite searches in both formal and grey literature, existing solutions fall 
short of addressing the specific challenges of the Linux kernel.
This research presents ArKanjo  , a command-line tool for Linux kernel maintenance designed to detect and
analyze function-level duplications. Released under the MIT
license, ArKanjo employs a two-stage architecture consisting
of a Preprocessor and a Query Responder that separates
computationally intensive analysis from efficient querying for duplications within large codebases.
Second, it analyzes the identified duplications to extract patterns 
and refactoring methods, offering guidance to future contributors. We evaluate ArKanjo 
against real world duplication cases in recent kernel versions, demonstrating
its effectiveness in identifying problematic clones that generic
tools often overlook. By identifying well-defined, manageable
duplication instances, ArKanjo effectively lowers the barrier
for new contributors, a capability evidenced by its role in guiding students to make their first 
code improvements to the kernel.

}
