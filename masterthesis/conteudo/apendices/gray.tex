\chapter{Related tools explored}

\label{app:gray}


This appendix contains the the list of related tools 
explored in this initial exploratory research mentioned
in the Introduction. To test this tools, we created a simple 
toy codebase with multiples duplicated code in C.

\paragraph{Copy/Paste Detector (CPD)} 

\

\textbf{Is free to use:} Yes.

We installed and tested it. It is a pretty primitive tool 
with little practical usage, and our experience using it 
was not good. It works by finding textual matches between files.

Access from \textit{\seqsplit{pmd.github.io/pmd/pmd\_userdocs\_cpd}}.

\paragraph{CppDepend}

\

\textbf{Is free to use:} No. There is a free trial periody.

Tools with paywalls do not meet our requirements, 
so we did not investigate the tool further for this reason.

Access from 
\textit{\seqsplit{cppdepend.com/blog/tracking-the-hidden-duplicate-code-in-a-c-code-base/}}.

\paragraph{Duplo}

\

\textbf{Is free to use:} Yes.

We installed and tested it. It is a pretty primitive tool 
with little practical usage, and our experience using it was good. 
It works by finding textual matches between files.

Access from
\textit{\seqsplit{github.com/dlidstrom/Duplo}}.

\paragraph{Duploq}

\

\textbf{Is free to use:} Yes.

It is a front-end application for Duplo only, designed to 
enable a more intuitive and configurable user experience. 
We did not test it further, as it has the same capabilities 
as Duplo itself.

Access from
\textit{\seqsplit{github.com/duploq/duploq}}.

\paragraph{Simian} 

\

\textbf{Is free to use:} No. 
There is a license for academic research and education use.

Tools with paywalls do not meet our requirements, 
so we did not investigate the tool further for this reason.

Access from
\textit{\seqsplit{simian.quandarypeak.com/}}.

\paragraph{Code Climate}

\

\textbf{Is free to use:} No. 
There is a license for open source use.

Tools with paywalls do not meet our requirements, so this 
tool was initially skipped. Out of curiosity, we tried it 
ourselves, and it did not work correctly with our toy
codebase of code duplications.

Access from
\textit{\seqsplit{codeclimate.com/}}.

\paragraph{Visual Detection of Duplicated Code}

\

\textbf{Is free to use:} In principle, yes.

We found this tool in a paper and found it interesting. 
Unfortunately, we were not able to find an executable 
version of this tool for testing.

Access from
\textit{\seqsplit{www.researchgate.net/publication/2866183\_Visual\_Detection\_of\_Duplicated\_Code}}.

\paragraph{duplicate-code-detection-tool}

\

\textbf{Is free to use:} Yes.

It is the only tool we found that is not based on textual 
matching. It looks pretty good for small codebases but 
performs poorly when used in complex codebases. This tool 
provided significant inspiration for this work, as 
mentioned in Chapter \ref{cha:tool}.

Access from
\textit{\seqsplit{github.com/platisd/duplicate-code-detection-tool}}.


